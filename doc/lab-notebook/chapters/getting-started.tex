\chapter{Getting Started}
\label{chap:getting-started}

Welcome to \project{}!
\project{} aims to reduce the cost of training large scale machine learning
models.
Our system allows users to dynamically train NLP models using preemptable
servers, such as AWS spot instances.
Current systems only allow a static cluster configuraton, with developer added
check pointing for fault tolerance.

Our user facing code runs in Python, and in the source tree you should be able
to run the command:
\begin{lstlisting}
  python -m project_pactum --version
\end{lstlisting}
This displays your current version and ensures you have the basic dependencies
installed.

Our design includes a daemon that monitors AWS instances, and allows you to add
local instances for testing on the same machine.
For setup, create a file called \lstinline|settings.py| in the base repository
directory.
Here's an example of what it should contain:
\begin{lstlisting}
from pathlib import Path

AWS_ACCESS_KEY_ID = 'YOUR_AWS_ACCESS_KEY_ID'
AWS_SECRET_ACCESS_KEY = 'YOUR_AWS_SECRET_ACCESS_KEY'
AWS_REGION = 'us-east-1'
AWS_NAME = 'Jon'
AWS_AMI_ID = 'Project Pactum 0.1.0.dev123'

SSH_USERNAME = 'jon'
SSH_KEY = Path.home() / '.ssh' / 'id_rsa'

CUDA_HOME = Path('/opt/cuda')
DEEPSPEED_DIR = Path.home() / 'src' / 'ucla' / 'project-pactum' / 'external' / 'deepspeed'
\end{lstlisting}


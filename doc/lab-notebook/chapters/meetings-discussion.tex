\chapter{Meetings/Discussion}
\label{ch:meetings}

\textbf{In Depth Meeting - Wednesday, 01/13/21} \\
High level discussion topics related to dataset partitioning and Data Paralellism:
\begin{itemize}
    \item Data partitioning: Harry's region based partitiong proposal
    \begin{itemize}
        \item Given $r$ regions (partitions essentially) and $N$ nodes, each
          region gets $N/r$ workers.
        \item If region $r_i$ loses some workers and region $r_j$ still has all
          workers $\rightarrow$ transfer some workers from $r_j$ to $r_i$
        \item In each region $\rightarrow$ use the PyTorch Distributed Data Sampler
          to make sure each minibatch is not-overlapping
    \end{itemize}

    \item Data partitioning: Small chunk-based partitioning
    \begin{itemize}
        \item Divide the whole dataset up into small chunks (100MB for example)
        \item Store on S3
        \item When worker fails, distribute its chunks to all remaining workers
        \item When workers added redistribute extra chunks to maintain roughly equal
          workload
        \item \textbf{Question:} If we keep losing workers, how will we eventually run
          when there is more data than the workers can fit on their disks? (I guess
          this goes for any method of partitioning though...)
    \end{itemize}
\end{itemize}

\vspace{1em}
High level discussion related to exploring model parallelism (MP)
\begin{itemize}
    \item \textbf{Main Question:} How to ensure reliability when running MP on
      transient servers?
    \item \textbf{Suggested Solutions (Very high level):}
    \begin{itemize}
        \item Split model into smaller parts than necessary $\rightarrow$
          if minimum 3 GPUs required for model to fit, split model into 6 parts
        \item Run small copies redundantly on other GPUs $\rightarrow$ if we
          lose a worker with some portions of the model other GPUs will have that
          portion
        \item Use some form of \textbf{erasure coding} to reconstruct missing data?
        \begin{itemize}
            \item What would erasure coding look like in this context?
        \end{itemize}
    \end{itemize}
\end{itemize}

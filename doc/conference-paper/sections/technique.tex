\section{Technique}
\label{sec:technique}

We assume these large models have a repeating structure.
This makes the memory requirements and runtimes more uniform, since you don't
fuse operations between the repeated structures.

We take advantage of the warning provided by preemptable instances.
Our design is meant to continue computation without having to stop and recover
from a checkpoint.
However, we would check point, very infrequently, as a last resort.

Our design uses a \textit{virtual pipeline} to shift computation among nodes.
Consider running on a 12 node cluster in a $4 \times 3$ ($P \times D$)
configuration.
Assume each server can hold a maximum of 8 transformers.

\begin{tikzpicture}[
  server/.style={draw},
]
  \node [server] (s0) {$D_0R_{0-5}$};
  \node [server, right=1em of s0] (s1) {$D_0R_{6-11}$};
  \node [server, right=1em of s1] (s2) {$D_0R_{12-17}$};
  \node [server, right=1em of s2] (s3) {$D_0R_{18-23}$};
  \node [server, below of=s0] (s4) {$D_1R_{0-5}$};
  \node [server, right=1em of s4] (s5) {$D_1R_{6-11}$};
  \node [server, right=1em of s5] (s6) {$D_1R_{12-17}$};
  \node [server, right=1em of s6] (s7) {$D_1R_{18-23}$};
  \node [server, below of=s4] (s8) {$D_2R_{0-5}$};
  \node [server, right=1em of s8] (s9) {$D_2R_{6-11}$};
  \node [server, right=1em of s9] (s10) {$D_2R_{12-17}$};
  \node [server, right=1em of s10] (s11) {$D_2R_{18-23}$};

  \draw [->, -latex, shorten >=1pt, semithick]
    (s0) edge (s1)
    (s1) edge (s2)
    (s2) edge (s3)
    (s4) edge (s5)
    (s5) edge (s6)
    (s6) edge (s7)
    (s8) edge (s9)
    (s9) edge (s10)
    (s10) edge (s11);
\end{tikzpicture}

One of the servers die

\begin{tikzpicture}[
  server/.style={draw},
]
  \node [server] (s0) {$D_0R_{0-5}$};
  \node [server, right=1em of s0] (s1) {$D_0R_{6-11}$};
  \node [server, right=1em of s1] (s2) {$D_0R_{12-17}$};
  \node [server, right=1em of s2] (s3) {$D_0R_{18-23}$};
  \node [server, below of=s0] (s4) {$D_1R_{0-5}$};
  \node [server, right=1em of s4] (s5) {$D_1R_{6-11}$};
  \node [server, right=1em of s5] (s6) {$D_1R_{12-17}$};
  \node [server, right=1em of s6] (s7) {$D_1R_{18-23}$};
  \node [server, below of=s4] (s8) {$\mathbf{S_8}$ $D_2R_{0-5}$};
  \node [server, right=1em of s8] (s9) {$\mathbf{S_9}$ $D_2R_{6-11}$};
  \node [server, right=1em of s9, cross out, draw=red] (s10) {$\mathbf{S_{10}}$ $D_2R_{12-17}$};
  \node [server, right=1em of s10] (s11) {$\mathbf{S_{11}}$ $D_2R_{18-23}$};

  \draw [->, -latex, shorten >=1pt, semithick]
    (s0) edge (s1)
    (s1) edge (s2)
    (s2) edge (s3)
    (s4) edge (s5)
    (s5) edge (s6)
    (s6) edge (s7)
    (s8) edge (s9)
    (s9) edge (s10)
    (s10) edge (s11);
\end{tikzpicture}

We can then have the other servers in the pipeline to pick up the slack.

\begin{tikzpicture}[
  server/.style={draw},
]
  \node [server] (s0) {$D_0R_{0-5}$};
  \node [server, right=1em of s0] (s1) {$D_0R_{6-11}$};
  \node [server, right=1em of s1] (s2) {$D_0R_{12-17}$};
  \node [server, right=1em of s2] (s3) {$D_0R_{18-23}$};
  \node [server, below of=s0] (s4) {$D_1R_{0-5}$};
  \node [server, right=1em of s4] (s5) {$D_1R_{6-11}$};
  \node [server, right=1em of s5] (s6) {$D_1R_{12-17}$};
  \node [server, right=1em of s6] (s7) {$D_1R_{18-23}$};
  \node [server, below of=s4, xshift=1em] (s8) {$\mathbf{S_8}$ $D_2R_{0-7}$};
  \node [server, right=1em of s8] (s9) {$\mathbf{S_9}$ $D_2R_{8-15}$};
  \node [server, right=1em of s9] (s11) {$\mathbf{S_{11}}$ $D_2R_{16-23}$};

  \draw [->, -latex, shorten >=1pt, semithick]
    (s0) edge (s1)
    (s1) edge (s2)
    (s2) edge (s3)
    (s4) edge (s5)
    (s5) edge (s6)
    (s6) edge (s7)
    (s8) edge (s9)
    (s9) edge (s11);
\end{tikzpicture}

$\mathbf{S_9}$ $D_2R_{6-7} \rightarrow \mathbf{S_8}$

$\mathbf{S_{10}}$ $D_2R_{12-15} \rightarrow \mathbf{S_9}$

$\mathbf{S_{10}}$ $D_2R_{15-17} \rightarrow \mathbf{S_{11}}$

This has a bit of a bottleneck, $\mathbf{S_9}$ sends 2 repeated structures, and
recieves 4.
$\mathbf{S_8}$ and $\mathbf{S_{11}}$ both recieve 2.
$\mathbf{S_{10}}$ has to send all 6 (unavoidable).
